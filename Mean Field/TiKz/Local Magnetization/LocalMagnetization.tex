\documentclass{standalone}
\usepackage{pgfplots}

\pgfplotsset{compat=1.18}

\begin{document}

\begin{tikzpicture}
    \begin{axis}[
            title={Sem Campos Magnéticos Aleatórios},
            xlabel={$h/|\mathcal{J}|$},
            ylabel={$\mathcal{M}$},
            width=1\textwidth % Define a largura do gráfico
        ]
        % First plot
        \addplot[
            color=black,
            dotted,
            line width=2pt,
        ]
        table[
                x index=0,
                y index=1,
                col sep=space
            ] {/home/mateus/Code/Fortran/Cluster/Mean Field/Data/NoRF.dat};

        % Second plot
        \addplot[
            color=blue,
            dotted,
            line width=2pt,
        ]
        table[
                x index=0,
                y index=2,
                col sep=space
            ] {/home/mateus/Code/Fortran/Cluster/Mean Field/Data/NoRF.dat};
        % Third plot
        \addplot[
            color=red,
            dotted,
            line width=2pt,
        ]
        table[
                x index=0,
                y index=3,
                col sep=space
            ] {/home/mateus/Code/Fortran/Cluster/Mean Field/Data/NoRF.dat};

    \end{axis}
\end{tikzpicture}

\hspace{1cm} % Espaçamento horizontal entre os gráficos

\begin{tikzpicture}
    \begin{axis}[
            title={Bimodal},
            xlabel={h/$|\mathcal{J}|$},
            ylabel={$\mathcal{M}$},
            width=1\textwidth % Define a largura do gráfico
        ]
        % First plot
        \addplot[
            color=black,
            dotted,
            line width=2pt,
        ]
        table[
                x index=0,
                y index=1,
                col sep=space
            ] {/home/mateus/Code/Fortran/Cluster/Mean Field/Data/Bi.dat};

        % Second plot
        \addplot[
            color=blue,
            dotted,
            line width=2pt,
        ]
        table[
                x index=0,
                y index=2,
                col sep=space
            ] {/home/mateus/Code/Fortran/Cluster/Mean Field/Data/Bi.dat};
        % Third plot
        \addplot[
            color=red,
            dotted,
            line width=2pt,
        ]
        table[
                x index=0,
                y index=3,
                col sep=space
            ] {/home/mateus/Code/Fortran/Cluster/Mean Field/Data/Bi.dat};

    \end{axis}
\end{tikzpicture}

\hspace{1cm} % Espaçamento horizontal entre os gráficos

\begin{tikzpicture}
    \begin{axis}[
            title={Trimodal},
            xlabel={h/$|\mathcal{J}|$},
            ylabel={$\mathcal{M}$},
            legend pos=outer north east,
            width=1\textwidth % Define a largura do gráfico
        ]

        % First plot
        \addplot[
            color=black,
            dotted,
            line width=2pt,
        ]
        table[
                x index=0,
                y index=1,
                col sep=space
            ] {/home/mateus/Code/Fortran/Cluster/Mean Field/Data/Tri.dat};
        \addlegendentry{Plot 1}

        % Second plot
        \addplot[
            color=blue,
            dotted,
            line width=2pt,
        ]
        table[
                x index=0,
                y index=2,
                col sep=space
            ] {/home/mateus/Code/Fortran/Cluster/Mean Field/Data/Tri.dat};
        \addlegendentry{Plot 2}

        % Third plot
        \addplot[
            color=red,
            dotted,
            line width=2pt,
        ]
        table[
                x index=0,
                y index=3,
                col sep=space
            ] {/home/mateus/Code/Fortran/Cluster/Mean Field/Data/Tri.dat};
        \addlegendentry{Plot 3}

    \end{axis}

\end{tikzpicture}


\end{document}
